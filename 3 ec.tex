We introduce equivariant cohomology and some classical results about the
$T$-equivariant cohomology of smooth projective toric varieties.

\subsection{Basic properties}
Let $G$ be a Lie group. The equivariant cohomology ring $H_G^*(X)$ of a $G$-space $X$ is
defined as the singular cohomology of the Borel construction $X\times_G EG$,
where $EG$ is a contractible space on which $G$ acts freely. Such a space
always exists and is unique up to homotopy equivalence \cite{hatcher}.

\hfill

Here are some important properties to know about equivariant cohomology:

\begin{enumerate}
	\item functoriality;
	\item a ring structure;
	\item excision;
	\item the Mayer-Vietoris sequence;
	\item the Künneth formula;
	\item the Leray spectral sequence;
	\item for smooth orientable $X$, Poincaré duality; and
	\item existence of Chern classes,
\end{enumerate}
where all subsets and maps are assumed to be equivariant. The
ring $H^*_T(X)$ is a module over $H^*_T(\text{pt})$ via the map $X \to \text{pt}$.

\hfill

We now introduce a notion of equivariant formality which is a particularly
nice algebraic notion.
It implies that
the equivariant cohomology of $X$ is a free module over the equivariant cohomology
of a point.

\begin{definition}
	A $G$-space $X$ is called \textbf{equivariantly formal} if the Leray spectral sequence
	associated to the fibration $X \to X \times_T ET \to BT$ collapses at the $E_2$-page.
\end{definition}

By definition, we have that
\begin{proposition}
	If $X$ is equivariantly formal, then \begin{align*}
		H^*_G(X) \cong H^*_G(\text{pt}) \otimes H^*(X).
	\end{align*}
\end{proposition}

When $X$ is equivariantly formal,
the ordinary cohomology can be recovered from
equivariant cohomology as the quotient
\[
	H^*(X) = \frac{H^*_T(X)}{M \cdot H^*_T(X)}
\]
which in effect simply sets each $t_i = 0$.
In particular, $H^*_T(X)$ is a free module over $H^*_T(\text{pt})$.

\hfill

Many varieties of interest are equivariantly formal, such as all of the following:
\begin{enumerate}
	\item a smooth complex projective variety (with respect to any linear algebraic $T$-action);
	\item a variety whose ordinary cohomology vanishes in odd degree (with respect to any $T$-action);
	\item a compact symplectic manifold with a Hamiltonian $T$-action, where $T$ is a compact torus.
\end{enumerate}
See \cite{fulton-anderson} for a good discussion of these facts.

\subsection{Examples}
We introduce some examples.

\begin{example}
	We can identify $U(n)$ as those complex matrices preserving
	the standard Hermitian form on $\C^n$. The group $U(n)$ acts on $S^{2n-1}$ transitively
	and the stabilizer of the point $(1,0,\ldots,0)$ is $U(n-1)$.

	Therefore there is a canonical action of $U(1)$ acting as scalar matrices on $S^{2n-1}$
	inherited from the action on $U(n)$. None of these odd dimensional spheres are contractible,
	but $S^\infty$ is contractible. In particular $EU(1) = S^\infty$ and \begin{align*}
		BU(1) = S^\infty/U(1) = \C\P^\infty
	\end{align*}
	Therefore \begin{align*}
		H^*_{U(1)}(pt) = H^*(\C\P^\infty) = \Z[t]
	\end{align*} where $t$ is the first Chern class of the tautological line bundle and $\deg t = 2$.
\end{example}

\begin{example}
	In general, $H_T^*(pt)$ can be identified with the representation ring of $T$. \begin{align*}
		BT \cong \prod_{\rank T} \C\P^\infty
	\end{align*}

	Given a representation $V$ of $T$,
	we can form a vector bundle on the classifying space whose
	total Chern class is equal to the class of $V$ in the representation ring, c.f. \cite{huybrechts}.
\end{example}
\subsection{Danilov's theorem}
The geometry of a smooth projective toric variety $X$ is particularly special because
it is encoded in the combinatorics of the fan of $X$.
Danilov's theorem gives a presentation of the ordinary cohomology ring of $X$ in terms of the
combinatorics of the fan of $X$.
\begin{theorem}\label{thm:danilov}
	[Danilov]
	Let $X$ be a smooth projective toric variety. Then the ordinary cohomology ring
	of $X$ is \begin{align*}
		H^*(X)\cong \Q[\alpha_1,\ldots,\alpha_n]/(I+J)
	\end{align*} where $I$ is the ideal generated by the linear relations among the
	$\alpha_i$
	\begin{align*}
		\sum_{i=1}^n \langle v_i,u\rangle \alpha_i \in I
	\end{align*} for $u\in M$ and $v_1,\ldots,v_n$ the fundamental
	vectors of the rays of the fan of $X$,
	and $J$ is the ideal generated by the Stanley-Reisner relations. \begin{align*}
		\alpha_{i_1}\cdots\alpha_{i_k}\in J \text{ if } \{i_1,\ldots,i_k\} \text{ is not a cone of the fan of } X.
	\end{align*}
\end{theorem}

Danilov's original proof of the theorem is quite involved and invokes
a lot of intersection theory, but using facts from equivariant cohomology
and toric geometry, we can give a much simpler proof. In particular
smooth projective toric varieties are equivariantly formal. The proof of theorem
\ref{thm:danilov} will be deferrred to the next section where we
introduce the localization theorem. It is a good example of how the theory of equivariant
cohomology enriches ordinary cohomology.

\subsection{ABBV localization}
At its core, the ABBV localization formula arises from the principle that,
under certain conditions, integrals over a compact space with a torus action can be “localized”
to the fixed points of the action. This idea can be traced back to the stationary
phase approximation in physics, where integrals are approximated by contributions
from critical points. In the equivariant cohomology setting, the fixed points of the torus action play a similar role.
\begin{theorem}
	[Atiyah-Bott, Berline-Vergne] Suppose a torus $T$ of dimensionl $l$ acts on a
	compact oriented manifold $M$ with fixed point set $F$. If $\phi \in H^*_T(M)$, then
	\begin{align*}
		\int_M \phi = \sum_{x\in F} \frac{i^*_x\phi}{e_x}
	\end{align*} where $e_x$ is the equivariant Euler class of the normal bundle of $x$ in $M$.
\end{theorem}
In particular, we will apply this theorem when the fixed point set is a finite set of points,
in which case the localization formula becomes \begin{align*}
	\int_M \phi = \sum_{x\in F} \frac{\phi(x)}{e_x}
\end{align*} where $e_x$ is the equivariant Euler class of $T_xM$. In particular, $e_x$
is the product of the weights of the action of $T$ on $T_xM$. Moreover, the weights are
distinct because the fixed point set is isolated.

\hfill 

We will now consider the localization theorem in an example.
In general, \cite{tu} contains a proof of the ABBV localization formula 
for $S^1$-actions.

\begin{example}
    Consider the two sphere $S^2$ with real coordinates $(x,y,z)$ and the $S^1$-action \begin{align*}
        e^{i\theta}\cdot(x,y,z) = (x\cos\theta - y\sin\theta, x\sin\theta + y\cos\theta, z).
    \end{align*}
    which rotates the sphere about the $z$-axis. The fixed points of this action are the north and south poles.

    \hfill 

    The Cartan model of equivariant cohomology suggests that we can consider integrating an equivariant cohomology
    class over $S^2$ by integrating corresponding \emph{equivariant differential forms} over $S^2$. In particular
    if we consider the equivariant symplectic form $\alpha = \omega + 2\pi\mu$ where $\mu(x,y,z) = z$
    is the moment map, viewed as a degree $0$ form. $\alpha$ is equivariantly closed because
    in the Cartan model, the equivariant differential is given by \begin{align*}
        d_X = d - ui_X
    \end{align*} where $i_X$ is the contraction with the vector field $X$ generating the $S^1$-action
    and $u$ is the standard representation of $S^1$, the "equivariant parameter". 

    Therefore $\alpha$ represents a class in $H^2_{S^1}(S^2)$. We can compute the integral of $\alpha$ over $S^2$ 
    using the localization formula. \begin{align*}
        \int_{S^2} \alpha &= \frac{\mu(N)}{\text{weight of $U(1)$ action on $T_N S^2$}} 
        + \frac{\mu(S)}{\text{weight of $U(1)$ action on $T_S S^2$}} \\
        &= 2\pi\mu(N) - 2\pi\mu(S) = 4\pi
    \end{align*} where $N$ and $S$ are the north and south poles respectively. Indeed
    the surface area of $S^2$ is $4\pi$.
\end{example}

\subsection{Proof of Danilov's theorem}
\subsubsection{Cone-orbit correspondence}
Let $T$ be an $n$-dimensional torus with character group $M$. Let $N = \Hom(M,\Z)$ be the
dual lattice, their pairing is denoted by $\langle \cdot,\cdot\rangle$. Let $X = X(\Sigma)$ be a smooth
complete toric variety, which are in bijection with complete nonsingular fans $\Sigma$ in $N_\R$.

\hfill

For any convex cone $\sigma \subset N_\R$, the \emph{dual cone} in $M_\R$ is \begin{align*}
    \sigma^\vee = \{u\in M_\R \mid \langle u,v\rangle \geq 0 \text{ for all } v\in \sigma\}.
\end{align*} By intersecting with the lattice, we obtain a semigroup $\sigma^\vee \cap M$
with corresponding semigroup algebra $\C[\sigma^\vee \cap M]$. The toric variety $X$
is covered by $T$-variant open affine sets \begin{align*}
    U_\sigma = \Spec \C[\sigma^\vee \cap M]
\end{align*} The affine charts corresponding to the top-dimensional cones of $\Sigma$ 
are enough to cover $X$, and the intersection of cones corresponds to the intersection of
affine charts.

\hfill

Each cone $\tau$ of the fan also defines a torus-invariant subvariety $V(\tau)$ of $X$ of codimension $\dim \tau$.
On open affines, the subvariety looks like \begin{align*}
    V(\tau) \cap U_\sigma = \Spec \C[\tau^{\perp} \cap \sigma^\vee \cap M] \hookrightarrow \Spec \C[\sigma^\vee \cap M]
\end{align*} and so elements of the dual lattice $N$ can be thought of as rational functions on $X$.
\subsubsection{Bialynicki-Birula decomposition}
The Bialynicki-Birula decomposition is a 
generalization of the Morse theory for torus actions. Suppose 
that $\C^*$ acts on a smooth projective variety $X$ with finitely 
many fixed points $p_1,\ldots,p_k$. 

\hfill

Then each $T_{p_i}X$ is a representation of $\C^*$, and so we can decompose
into weight spaces \begin{align*}
    T_{p_i}X = \bigoplus_{\lambda\in \C} V_\lambda
\end{align*} where $V_\lambda = \set{v\in T_{p_i}X \mid t\cdot v = t^\lambda v}$.
Note that $\lambda\neq 0 $ because the fixed point set is isolated.

Define the attracting set \begin{align*}
    C_i = \set{x\in X \mid \lim_{t\to 0} t\cdot x = p_i}
\end{align*} 

\begin{theorem}
    [Bialynicki-Birula] 
    There existsa filtration of $X$ by closed subschemes \begin{align*}
        X = X_n \supset X_{n-1} \supset \cdots \supset X_0 
    \end{align*} such that 
    each $X_i \backslash X_{i-1}$ is a disjoint union of affine spaces called cells, 
    in particular the attracting sets.
\end{theorem}

\begin{cor}
    Let $X$ as above. Then \begin{enumerate}
        \item $H_{2i+1}(X) = 0$ for all $i$;
        \item $H_{2i}(X)$ is a $\Z$-module freely generateed by the classes
        of the closures of the $i$-dimensional cells.
    \end{enumerate}
\end{cor}

\subsubsection{Shellings}
In this section, we apply the setup 
of the Bialynicki-Birula decomposition to the context of toric varieties.

\hfill

Let $X = X(\Sigma)$ be projective, with $P$ 
a polytope whose normal fan is $\Sigma$. Choosing
a general vector $v\in N_\R$ we obtain an ordering of the vertices 
$u_1,\ldots,u_s$ of $P$ by the order of the inner products $\langle v,u_i\rangle$.
Geometrically, we are choosing a $1$-parameter subgroup of the torus $T$ 
which acts on $X$. The correpsonding sub-moment map turns out to be
a perfect Morse-Bott function on $X$.

\hfill

By the polytope-fan correspondence,
we get an ordering of the maximal cones $\sigma_1,\ldots,\sigma_s$ of $\Sigma$.
For $1\leq i \leq s$ let \begin{align*}
    \tau_i = \bigcap_{j>i, \dim(\sigma_j \cap \sigma_i) = n-1} \sigma_j \cap \sigma_i
\end{align*} so that 
$\tau_1 = \set{0}, \tau_s = \sigma_s$ and $\tau_p \subset \tau_q$
implies $p \leq q$. Such an ordering of cones 
is called a \emph{shelling} of $\Sigma$. 

\begin{proposition}
    A shelling gives a cellular decomposition of $X$, with
the closures of the cells being $V(\tau_i)$. In particular,
the classes \begin{align*}
    \alpha_i = [V(\tau_i)] \in H^{2(n-\dim \tau_i)}(X)
\end{align*}
form an additive $\Z$-basis of $H^*(X)$.
\end{proposition}


The $V(\tau_i)$ are precisely the closures of the attracting sets 
of the chosen $\C^*$-action on $X$. We will demonstrate this in a particularly
pleasant example.
\begin{example}
    [Morse theory on $\C\P^2$]
    Classically, recall that the Chow ring (or cohomology ring) of
     $\C\P^2$ is \begin{align*}
        H^*(\C\P^2) = \Z[0] \oplus \Z[\P^1] \oplus \Z[\P^2]
    \end{align*} Consider the standard action of $T^2$ on $\C\P^2$
    and consider the $1$-parameter subgroup acting by $t\cdot[x:y:z] = [tx:t^2y:z]$.
    Consider the following moment image, whose 
    edges are labled by the weights of the action on the tangent space at the fixed points:
    
\end{example}


\subsection{GKM Theory}
