We provide a brief introduction to connections on vector bundles. See \cite{huybrechts} for more details.
\subsection{Parallel transport}
One way to think about a connection is to consider parallel transport. You want to 
be able to differentiate sections of a vector bundle along paths. When we are dealing with functions,
we can form the directional derivative  \begin{align*}
    ds(x)X = \lim_{t\to 0}\frac{s(\gamma(t)) - s(\gamma(0))}{t}
\end{align*} for any smooth path $\gamma$ representing the tangent vector $X\in T_xM$ 
and this expression gives us a linear map $ds(x):T_xM\to E_x$.

However if $E$ is a nontrivial bundle, then this expression does not make sense because the
 summands live in different fibers. There is unfortunately no natural way to 
 compare vectors in different fibers. Therefore we need to introduce additional structure to 
 be able to compare these fibers.

For a general vector bundle $E\to M$, we want to associate, to a path $\gamma$ in $M$, a smooth family of 
parallel transport isomorphisms $P^t_\gamma: E_{\gamma(0)}\to E_{\gamma(t)}$ such that \begin{itemize}
    \item $P^0_\gamma = \id$
    \item $P^t_{\gamma_1\cdot\gamma_2} = P^t_{\gamma_2}\circ P^t_{\gamma_1}$
\end{itemize} for any paths $\gamma_1,\gamma_2$ and $t\in\R$.


Such a choice would allow us to define the directional ("covariant") derivative of a section $s$ along a path $\gamma$ 
as before. We shuold require that \begin{itemize}
    \item The directional derivative depends only on $s$ and $X\in T_xM$, not the particular 
    choice of $\gamma$.
    \item The map $\nabla s(x):T_xM\to E_x$ is $\C$-linear.
\end{itemize}
This will give us the richest notion of a connection on a vector bundle.
\subsection{Connections}
In this section, we consider $M$ a real manifold and $\pi:E\to M$ complex vector bundle.
Let $\cA^i(E) = \Omega^i(M) \otimes E$ denote the sheaf of smooth $i$-forms with values in $E$.
\begin{definition}
    A \textbf{connection} on $E$ 
    is a $\C$-linear map of sheaves
    $\nabla:\cA^0(E)\to\cA^1(E)$ satisfying the Leibniz rule
    \begin{align*}
        \nabla(fs) = df\otimes s + f\nabla s
    \end{align*}
\end{definition}
We can interpret this definition in the sense of parallel transport. Given 
a section $s\in\cA^0(E)$, we can differentiate it along a path $\gamma$ to get another 
section of $E$, i.e. $\nabla:\Gamma(E)\to \Gamma(\Hom(TM,E))$.
\begin{theorem}
    The space of all connections $\cA(E)$ is an affine space modelled on $\cA^1(\End E)$. In particular \begin{itemize}
        \item $\cA(E)$ is nonempty
        \item For any two connections $\nabla_1,\nabla_2$ the difference $\nabla_1-\nabla_2$ 
        is a global section of $\cA^1(\End E)$.
        \item $(\nabla + a)s := \nabla s + as$ is a connection whenever $\nabla$ is a connection and 
        $a\in\cA^1(\End E)$.
    \end{itemize}
\end{theorem}

The idea of a connection generalizes the exterior differential to sections of general vector bundles. 
However, a connection need not satisfy $\nabla^2 = 0$ in general. The obstruction for a connection
define a differential is measured by its curvature. We explain this now.

\subsection{Curvature}

A connection $\nabla:\cA^0(E)\to \cA^1(E)$ induces "differentials" \begin{align*}
    \nabla:\cA^i(E)\to\cA^{i+1}(E)
\end{align*} given by the formula \begin{align*}
    \nabla(\alpha\otimes s) = d\alpha\otimes s + (-1)^i\alpha\wedge\nabla s
\end{align*}

\begin{definition}
    The \textbf{curvature} $F_\nabla$ of a connection $\nabla$ is the composition \begin{align*}
        F_\nabla := \nabla^2:\cA^0(E)\to\cA^2(E)
    \end{align*} In particular $F_\nabla$ is a global section of $\cA^2(\End E)$. This is because 
    the curvature homomorphism is $\cA^0$-linear.
\end{definition}

\begin{example}
    Consider the connections on the trivial bundle $M\times \C^r$. If $\nabla = d$ is the
    trivial connection then $F_\nabla = 0$. 

    Any other connection is of the form $\nabla = d + A$ where $A$ is a matrix of 1-forms. For a section
    $s$ we compute \begin{align*}
        F_\nabla(s) &= (d+A)(d+A)(s) \\
        &= d^2s + dAs + Ads + AAs \\
        &= d(A)s + A\wedge As
    \end{align*} and therefore \begin{align*}
        F_\nabla = dA + A\wedge A
    \end{align*}
    For line bundles we get that $F_\nabla = dA$ is an ordinary 2-form.
\end{example}